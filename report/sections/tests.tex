\subsection{Tests}

Chacune des fonctions du driver, du firmware et communes au driver et au
firmware ont été testé. Le protocole de communication entre le driver et le
firmware a également été testé au moyen d'une communication entre processus
avant de mettre en place le code sur les cartes.

Les programmes de tests se trouvent dans des sous dossiers \emph{tests}.
Chacun des composants (driver, firmware et commun) a doit à son dossier de
test. La plupart des tests affichent le résultat effectif comparé au résultat
attendu, cela permet notamment de voir si un paquet formé est différent de ce
qu'on aurait du obtenir et de voir sur quelle partie du paquet il y a une
erreur.

D'autres tests utilisent des assertions afin de vérifier plus rapidement la
validité des fonctions utilisées. C'est notamment le cas du test de
communication dans lequel le driver envoie une commande au firmware qui
répond ; une assertion est alors faite sur la validité de la réponse par
rapport à ce qui était attendu.

\subsection{Benchmark}

% TODO
