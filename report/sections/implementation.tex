Le driver et le firmware sont tous les deux écrits en C. Par conséquent, le
code qui pouvait être mis en commun l'a été tant que possible. Cela concerne
notamment la gestion des masks, du failsafe, la fabrication de paquet, ainsi
que la définition du protocole.

De plus, afin d'avoir un code du firmware sensiblement identique, que le code
soit embarqué ou non, une macro de préprocesseur est utilisée. Lorsque
\emph{EMBEDDED} est définie le code est compilé pour fonctionner sur
l'arduino. Inversement, sans la définition de cette macro le code peut-être
utilisé pour effectuer des tests de communication interprocessus afin de
tester le protocole.

%TODO: détailler ce qui change concrètement entre l'utilisation dans l'arduino
%ou non

\section{Driver}

\section{Firmware}

