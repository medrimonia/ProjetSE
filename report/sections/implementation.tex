\paragraph{}
Le driver et le firmware sont tous les deux écrits en C. Par conséquent, le
code qui pouvait être mis en commun l'a été tant que possible. Cela concerne
notamment la gestion des masks, du failsafe, la fabrication de paquet, ainsi
que la définition du protocole.

\paragraph{}
De plus, afin d'avoir un code du firmware sensiblement identique, que le code
soit embarqué ou non, une macro de préprocesseur est utilisée. Lorsque
\verb!EMBEDDED! est définie le code est compilé pour fonctionner sur
l'arduino. Inversement, sans la définition de cette macro le code peut-être
utilisé pour effectuer des tests de communication interprocessus afin de
tester le protocole.

%TODO: détailler ce qui change concrètement entre l'utilisation dans l'arduino
%ou non

\subsection{Makefile}
% Cibles du Makefile
\paragraph{}
Les différents binaires que nous cherchions à produire étaient les suivants~:
\begin{itemize}
\item Firmware
\item Fake-firmware
      \footnote{Une version du firmware à employer en local sur un ordinateur
                pour permettre de valider la communication}
\item Différents binaires de tests
\end{itemize}

% Multi-cible : un Makefile
\paragraph{}
Nous avons souhaité les produire tous à l'aide d'un unique Makefile, car, tous
les exécutables partageant du code, cela nous permettait d'assurer facilement
que nous n'avions pas détruit la compilation des autres binaires en effectuant
des modifications. Cet aspect était rendu encore plus important par la présence
des flags de compilation risquant d'engendrer des situations ou une modification
apporte des problèmes de compilation dans un autre mode que celui qui est
étudié.

% Distinction de sources
\paragraph{}
Bien qu'une grande partie du code soit commune, il existe toute de même des
différences. Celles-ci se sont traduites sous la forme de trois catégories
séparées~:
le code commun, le code spécifique au driver et le code spécifique au firmware.
Afin d'éviter d'utiliser des dépendances sans s'en rendre compte, des règles
spécifique à chaque catégorie ont été créé, permettant de s'assurer que seuls
les headers acceptable pour la catégorie étaient utilisés\footnote{La partie
commune ne devait dépendre ni de celle du driver, ni de celle du firmware. Celle
du firmware ne devait pas dépendre du driver et inversément}.

% Distinction de cibles
\paragraph{}
Outre la différence provenant de l'architecture logicielle, il existe aussi une
différence de cible. Effectivement, le {\em fake-firmware} et le {\em firmware}
correspondent aux mêmes objets, mais compilés de manière différente. Par
conséquent, il nous aurait semblé étrange d'utiliser plusieurs Makefile pour
effectuer la même tâche. Afin de résoudre le problèmes de la nécessité d'avoir
des objets compilé avec différentes options, nous avons décidé de placer ceux
destinés à une utilisation locale et ceux destinés à une utilisation embarquée
dans deux dossiers différents.

\subsection{Driver}

\subsection{Firmware}

\subsubsection{Gestion des registres}

\paragraph{}
Pour la réalisation de la communication UART il nous a fallu décider de la façon
de récuperer les informations de la communication UART. Pour cela nous avons
décidé d'utiliser les interruptions de la communication UART. Cependant avant de
lire ou d'écrire sur les lignes de communications, il nous a fallu initialiser
la communication

\paragraph{}
Pour effectuer cela, il a été nécessaire de chercher dans la datasheet, les
différents registres à initialiser. Ces registres sont au nombre de trois:
\begin{itemize}
  \item Le registre \verb!UBRR! qui contient la valeur de la vitesse de communication
    fonction de la fréquence du micro-contrôleur et du baudrate suivant la
    formule suivante:\begin{equation}
      UBRR = \frac{f_{OSC}}{16 \cdot Baudrate} -1
      \label{baudrate}
    \end{equation}
  \item Les registres \verb!UCSRA! et \verb!UCSRB! qui sont les registres de contrôle et de
    statut de la communication.
  \item Le registre \verb!UCSRC! qui sert à définir le format de la trame
\end{itemize}

\paragraph{}
Après cette initialisation il était nécessaire de connaitre la façon de
récupérer et d'envoyer un message à partir du bus de communication. Pour cela
nous sommes retournés à la recherche des registres concernés dans la datasheet.
Après recherche nous sommes arrivé au point que la donnée reçue / envoyée est
stocké avant traitement dans le registre \verb!UDR!. Ce registre admet deux
comportements différents en fonction de la façon dont il est utilisé. Dans le
cas d'une lecture, le registre retournera la valeur sur 8bit du récepteur. Dans
le cas d'une écriture , la valeur stockée par le registre sera envoyée lorsque
le \verb!Transmit Shift Register! sera vide.


\subsubsection{Taille d'exécutable}
% Confrontation au problème de la taille
\paragraph{}
Après avoir malencontreusement écrasé notre bootloader alors que nous pensions
que la taille de notre exécutable était raisonnable, il est devenu flagrant que
réduire la taille de celui-ci était une priorité.

% Essai avec des connaissances lacunaires
\paragraph{}
Armé de nos connaissances lacunaires dans ce domaine, nous avons cherché à
réduire la taille du binaire obtenu. Après avoir essayé de nombreuses
{\em astuces} de compilation trouvées sur internet, nous n'avons pas réussi à
avoir le moindre gain. Nous avons trouvé d'autre conseils qui indiquaient qu'il
valait mieux utiliser autant que possible des entiers de taille définie
\verb!int8_t! ou \verb!int16_t! plutôt que des entiers simples \verb!int!.

% Gain avec les entiers de taille donnée
\paragraph{}
Passer l'entièreté de notre projet au crible de cette règle nous a permis
d'obtenir un léger gain
          \footnote{Nous sommes passé d'environ 7000 bytes à 6850.}.
Cependant, il nous est vite apparu que ces efforts ne suffiraient pas à obtenir
de gain significatif.

% Flinguage du failsafe
\paragraph{}
Afin d'être sûr de pouvoir tester notre firmware sur la carte dans le temps
imparti, nous avons donc décidé d'opter pour une solution brutale mais dont
l'efficacité ne faisait aucun doute. Nous avons ajouté un flag de compilation
\verb!DISABLE_FAILSAFE! permettant de désactiver tout le code lié à cette
fonctionnalité.
