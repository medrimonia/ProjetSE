\paragraph{}
Au cours de ce projet, nous avons fait face à différentes difficultés, allant
de la difficulté d'établir un protocole commun aux problèmes de taille
d'exécutable.

% Difficulté d'établissement du protocole
\paragraph{}
Au cours du projet, nous avons pu voir les difficultés inhérentes à la
spécification d'un protocole. Effectivement, la version que nous avons établie
en groupe laissait trop de blancs dans les spécifications. Même dans la
version qui nous a été proposée ensuite, il restait quelques points de détails
à clarifier. Cela nous a montré qu'il est nécessaire que les protocoles
soient implémentés et testés par différentes personnes afin de les éprouver.

% Pénibilité d'implémentation
\paragraph{}
Au fur et à mesure de l'implémentation, nous avons pu prendre conscience des
difficultés de débogage d'un protocole qui cherche à travailler au bit près.
De plus, l'écriture de ce code nous a été pénible de par sa répétitivité. Nous
avons donc pu conclure que si la nécessité d'optimiser les protocoles déjà
existant n'est pas établie, il vaut mieux utiliser les solutions existantes
afin d'éviter trop de complications.

% De l'amélioration de la latence
\paragraph{}
Il était prévu de définir une mesure de la latence afin d'effectuer des
benchmarks et optimiser les performances de la communication. Cependant, du
fait des problèmes successifs que nous avons eu avec nos cartes et du temps
passé sur l'écriture de l'API nous n'avons malheureusement pas pu traiter
cette partie du projet.

% Une API incomplète
\paragraph{}
Enfin, l'API - bien que le code soit de taille conséquente - ne traite pas
l'ensemble des erreurs pouvant survenir dans la communication. Par exemple le
protocole permet d'avoir un code de retour signalant une erreur dans les
paquets de réponse ; ce code de retour n'est pas toujours pris en compte par
notre API. Avec du temps supplémentaire nous aurions pu améliorer la prise en
charge des erreurs dans la communication.
