Ce projet a pour but de standardiser la communication entre la Raspberry Pi et
l'ATmega8 en établissant un protocole à la fois flexible - pour permettre de
futures évolutions et de nouvelles version - et performant afin d'améliorer la
latence lors de la communication entre les deux cartes.

Le projet a été découpé en plusieurs étapes : tout d'abord l'établissement
d'un protocole commun entre les différents groupes, puis la définition d'une
API implémentant ce protocole en respectant les conditions de flexibilité et
de latence fixées. Enfin la dernière étape consistait à définir une mesure de
la latence afin d'effectuer des benchmarks et optimiser les performances de la
communication.

Cependant la Raspberry Pi n'acceptant pas des tensions supérieures à 3,3V , on
sera donc obligé de conserver le passage par le bus USB pour obtenir une
communication entre les deux cartes.
