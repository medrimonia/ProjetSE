\subsection{Arduino}

  % Pour l'arduino , on a le microcontrolleur qui va gérer la communication et les pinouilles de PWM et de gestion des pinouilles analogiques.Du
  % coup il faut être capable à tout instant de pouvoir recevoir les informations de la carte intelligence (la raspberry) -> d'où les
  % interruptions. Seul problème du coup , on est pas sur que on va pas surcharger avec la communication le temps de calcul du microP et que du
  % coup il pourra plus activer la PWM au temps où il faut. Après observation du soft arduino , les interruptions sont interdites par le soft pour
  % être sur que le microcontrolleur puisse générer, dans les temps , chaque signal. Cela pose donc le problème de création d'un scheduler
  % permettant la périodisation des taches, cependant on s'est rendu compte après compilation du code que le binaire généré atteint déja les
  % limites de possibilité de l'atmega. La création d'un scheduler est donc impossible actuellement, il faudrait pour cela réduire le binaire et
  % ensuite s'assurer que la communication avec la Raspberry Pi ne parasitera pas par son temps d'occupation du micro-contrôleur  la génération de
  % signaux.


  L'arduino est actuellement, une solution rapide et fiable d'utiliser une carte
électronique pour développer un projet.
\paragraph{}
  Dans le cas de ce projet, la carte utilisée est une DAGU Mini Driver Control
Board, qui embarque un ATmega8. Elle est dotée de deux bus de communications: un
bus SPI et un bus UART.
\paragraph{}
  Ce dernier va être utilisé pour la communication avec une Raspberry Pi. Cette
communication étant asynchrone full duplex, il va falloir une façon de
déterminer l'arrivée d'une nouvelle donnée. Pour cela les
vecteurs d'interruption fournit avec le micro-contrôleur vont être utilisé.
Cependant en fonctionnant comme cela, le soft s'interdit l'utilisation de l'API
arduino pour la génération de PWM 8 et 16 bits.
\paragraph{}
  L'explication de cette limite s'explique par le fait que l'API Arduino utilise
une boucle pour appliquer ses différents signaux et ainsi si une interruption
apparait la valeur de la PWM sera faussée. Ainsi pour être sûr que la PWM
générée est de la bonne valeur, il faudrait utiliser le système interne du
micro-contrôlleur qui génère automatiquement les PWM si le timer est
convenablement initialisé. Cependant pour des problèmes de taille de binaire,
il est actuellement impossible de rajouter de nouvelles gestions de PIN ou
encore un scheduler basé sur un timer avec une priorité sur l'interruption.
\paragraph{}
  La solution serait de réduire la taille du binaire pour s'assurer de ne pas
réécrire sur le bootloader et ensuite s'assurer que la communication avec la
Raspberry Pi ne parasitera pas, par son temps d'occupation du
micro-contrôleur, la génération de signaux.\\


\subsection{RaspberryPi}
